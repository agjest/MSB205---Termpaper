% Options for packages loaded elsewhere
\PassOptionsToPackage{unicode}{hyperref}
\PassOptionsToPackage{hyphens}{url}
%
\documentclass[
]{article}
\title{ag comments}
\author{}
\date{\vspace{-2.5em}}

\usepackage{amsmath,amssymb}
\usepackage{lmodern}
\usepackage{iftex}
\ifPDFTeX
  \usepackage[T1]{fontenc}
  \usepackage[utf8]{inputenc}
  \usepackage{textcomp} % provide euro and other symbols
\else % if luatex or xetex
  \usepackage{unicode-math}
  \defaultfontfeatures{Scale=MatchLowercase}
  \defaultfontfeatures[\rmfamily]{Ligatures=TeX,Scale=1}
\fi
% Use upquote if available, for straight quotes in verbatim environments
\IfFileExists{upquote.sty}{\usepackage{upquote}}{}
\IfFileExists{microtype.sty}{% use microtype if available
  \usepackage[]{microtype}
  \UseMicrotypeSet[protrusion]{basicmath} % disable protrusion for tt fonts
}{}
\makeatletter
\@ifundefined{KOMAClassName}{% if non-KOMA class
  \IfFileExists{parskip.sty}{%
    \usepackage{parskip}
  }{% else
    \setlength{\parindent}{0pt}
    \setlength{\parskip}{6pt plus 2pt minus 1pt}}
}{% if KOMA class
  \KOMAoptions{parskip=half}}
\makeatother
\usepackage{xcolor}
\IfFileExists{xurl.sty}{\usepackage{xurl}}{} % add URL line breaks if available
\IfFileExists{bookmark.sty}{\usepackage{bookmark}}{\usepackage{hyperref}}
\hypersetup{
  pdftitle={ag comments},
  hidelinks,
  pdfcreator={LaTeX via pandoc}}
\urlstyle{same} % disable monospaced font for URLs
\usepackage[margin=1in]{geometry}
\usepackage{color}
\usepackage{fancyvrb}
\newcommand{\VerbBar}{|}
\newcommand{\VERB}{\Verb[commandchars=\\\{\}]}
\DefineVerbatimEnvironment{Highlighting}{Verbatim}{commandchars=\\\{\}}
% Add ',fontsize=\small' for more characters per line
\usepackage{framed}
\definecolor{shadecolor}{RGB}{248,248,248}
\newenvironment{Shaded}{\begin{snugshade}}{\end{snugshade}}
\newcommand{\AlertTok}[1]{\textcolor[rgb]{0.94,0.16,0.16}{#1}}
\newcommand{\AnnotationTok}[1]{\textcolor[rgb]{0.56,0.35,0.01}{\textbf{\textit{#1}}}}
\newcommand{\AttributeTok}[1]{\textcolor[rgb]{0.77,0.63,0.00}{#1}}
\newcommand{\BaseNTok}[1]{\textcolor[rgb]{0.00,0.00,0.81}{#1}}
\newcommand{\BuiltInTok}[1]{#1}
\newcommand{\CharTok}[1]{\textcolor[rgb]{0.31,0.60,0.02}{#1}}
\newcommand{\CommentTok}[1]{\textcolor[rgb]{0.56,0.35,0.01}{\textit{#1}}}
\newcommand{\CommentVarTok}[1]{\textcolor[rgb]{0.56,0.35,0.01}{\textbf{\textit{#1}}}}
\newcommand{\ConstantTok}[1]{\textcolor[rgb]{0.00,0.00,0.00}{#1}}
\newcommand{\ControlFlowTok}[1]{\textcolor[rgb]{0.13,0.29,0.53}{\textbf{#1}}}
\newcommand{\DataTypeTok}[1]{\textcolor[rgb]{0.13,0.29,0.53}{#1}}
\newcommand{\DecValTok}[1]{\textcolor[rgb]{0.00,0.00,0.81}{#1}}
\newcommand{\DocumentationTok}[1]{\textcolor[rgb]{0.56,0.35,0.01}{\textbf{\textit{#1}}}}
\newcommand{\ErrorTok}[1]{\textcolor[rgb]{0.64,0.00,0.00}{\textbf{#1}}}
\newcommand{\ExtensionTok}[1]{#1}
\newcommand{\FloatTok}[1]{\textcolor[rgb]{0.00,0.00,0.81}{#1}}
\newcommand{\FunctionTok}[1]{\textcolor[rgb]{0.00,0.00,0.00}{#1}}
\newcommand{\ImportTok}[1]{#1}
\newcommand{\InformationTok}[1]{\textcolor[rgb]{0.56,0.35,0.01}{\textbf{\textit{#1}}}}
\newcommand{\KeywordTok}[1]{\textcolor[rgb]{0.13,0.29,0.53}{\textbf{#1}}}
\newcommand{\NormalTok}[1]{#1}
\newcommand{\OperatorTok}[1]{\textcolor[rgb]{0.81,0.36,0.00}{\textbf{#1}}}
\newcommand{\OtherTok}[1]{\textcolor[rgb]{0.56,0.35,0.01}{#1}}
\newcommand{\PreprocessorTok}[1]{\textcolor[rgb]{0.56,0.35,0.01}{\textit{#1}}}
\newcommand{\RegionMarkerTok}[1]{#1}
\newcommand{\SpecialCharTok}[1]{\textcolor[rgb]{0.00,0.00,0.00}{#1}}
\newcommand{\SpecialStringTok}[1]{\textcolor[rgb]{0.31,0.60,0.02}{#1}}
\newcommand{\StringTok}[1]{\textcolor[rgb]{0.31,0.60,0.02}{#1}}
\newcommand{\VariableTok}[1]{\textcolor[rgb]{0.00,0.00,0.00}{#1}}
\newcommand{\VerbatimStringTok}[1]{\textcolor[rgb]{0.31,0.60,0.02}{#1}}
\newcommand{\WarningTok}[1]{\textcolor[rgb]{0.56,0.35,0.01}{\textbf{\textit{#1}}}}
\usepackage{graphicx}
\makeatletter
\def\maxwidth{\ifdim\Gin@nat@width>\linewidth\linewidth\else\Gin@nat@width\fi}
\def\maxheight{\ifdim\Gin@nat@height>\textheight\textheight\else\Gin@nat@height\fi}
\makeatother
% Scale images if necessary, so that they will not overflow the page
% margins by default, and it is still possible to overwrite the defaults
% using explicit options in \includegraphics[width, height, ...]{}
\setkeys{Gin}{width=\maxwidth,height=\maxheight,keepaspectratio}
% Set default figure placement to htbp
\makeatletter
\def\fps@figure{htbp}
\makeatother
\setlength{\emergencystretch}{3em} % prevent overfull lines
\providecommand{\tightlist}{%
  \setlength{\itemsep}{0pt}\setlength{\parskip}{0pt}}
\setcounter{secnumdepth}{-\maxdimen} % remove section numbering
\ifLuaTeX
  \usepackage{selnolig}  % disable illegal ligatures
\fi

\begin{document}
\maketitle

\begin{itemize}
\item
  Dokumentet var reproduserbart. Flott!
\item
  pdf dokumentet er fint formatert. Bra
\item
  Dette kommer i reprise: «Når man samler inn data er det mest oppnåelig
  å ha et tilfeldig utvalg. Datainnsamlingen i hedonisk
  eiendomsverdi-studier fokuserer hovedsakelig på eneboliger.»
\item
  Grei gjennomgang av Bishop et al.
\item
  Ser dere bruker \texttt{attach()} mange mener dette bør ungås. Se
  \url{https://www.r-bloggers.com/2011/05/to-attach-or-not-attach-that-is-the-question/}.
  Så lenge dere jobber i \texttt{tidyverse} er det langt på veg
  unødvendig.
\item
  Fin gjennomgang av kc\_tracts10\_env\_data versus
  kc\_tracts10\_shore\_env\_var
\item
  Fin EDA fra Geoda, fin oppsummering av funn i EDA
\item
  Def. av mod2, mod3 etc. Lurt å bruke linjeskift i formula definisjonen
  så slipper dere at linjen går langt ut til høyre
\item
  Variablene sqft\_living15 og sqft\_lot15 burde dere trolig droppet.
  Som dere ser av dokumentasjonen er dette gjennomsnittet for de 15
  nærmeste naboene. Vi tar hensyn til dette ved hhv 3 og 10 naboer
\item
  Ad «\textbf{I .pdf-filen så har huxreg-tabellen lagt seg helt på
  slutten av dokumentet».} Når LaTeX ikke finner en god nok plassering
  av en figur eller tabell havner den i float og havner til slutt i
  dokumentet. Det er muligheter for et stykke på veg å presse LaTeX til
  å plassere figur/tabell der den er (kan kreve at man må inn i .tex
  filen). Ellers er en mulighet å bare la den flyte til slutten men
  bruke referanse/kryssreferanse.
\item
  En liten kommentar til lm plottene ville gjort seg. Ser at det ser ut
  til å være problem med heteroskedastisitet og også avvik fra
  normalitet på residualene. Kan gi problemer med statistisk inferens.
\item
  Fin test på tidsdummy
\item ~
  \hypertarget{moran-i-statistic-standard-deviate-28.301-p-value}{%
  \subsection{~Moran I statistic standard deviate = 28.301, p-value
  \textless{}}\label{moran-i-statistic-standard-deviate-28.301-p-value}}

  \#\# ~0.00000000000000022

  Ser her problemet med \texttt{options(scipen\ =\ 999)} , men vi lever
  i et fritt land ;-)
\item
  moran.plot(log(kc\_house\_data\_666\$price), listw =
  kc\_house\_data\_666\_W, labels Bruker dere linjeskift i koden blir
  den lettere å lese, f.eks

\begin{Shaded}
\begin{Highlighting}[]
\CommentTok{\#moran.plot(log(kc\_house\_data\_666$price), listw = kc\_house\_data\_666\_W, labels = FALSE, pch = 20, cex = 0.3)}
\FunctionTok{moran.plot}\NormalTok{(}
    \FunctionTok{log}\NormalTok{(kc\_house\_data\_666}\SpecialCharTok{$}\NormalTok{price), }
    \AttributeTok{listw =}\NormalTok{ kc\_house\_data\_666\_W, }
    \AttributeTok{labels =} \ConstantTok{FALSE}\NormalTok{, }
    \AttributeTok{pch =} \DecValTok{20}\NormalTok{, }
    \AttributeTok{cex =} \FloatTok{0.3}
\NormalTok{    )}
\end{Highlighting}
\end{Shaded}
\item
  Ok Lagrange multiplikator test
\item
  Grei argumentasjon for «local». Kunne event gitt en referanse til
  LeSage 2014
\item
  Impacts for SDEM flott gjort
\item
  Fin diskusjon romlige effekter i residualen mod2 versus mod3
\item
  Grei konklusjon
\item
  Litt mer inngående diskusjon av koeffisientene i Full, 2000 Seed og
  666 Seed hadde vært en fordel. Er de konsistente eller skifter de
  kraftig far modell til modell.\\
\end{itemize}

\end{document}
